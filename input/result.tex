\begin{itemize}
\item The results of the project work have been depicted in the Figure below. The results depict that almost ninety percent of the time is being saved when we use automation. This not only helps to increase efficiency but also enhances the work experience of the user.
\item The graphics used in the application are made using GIMP v2.6 which supports batch processing of images and thus acts as a great added advantage when several image files have to be processed simultaneously. This saves time and effort required to modify each file individually. Also by setting the properties of the various layers of the images during making the graphics for the souvenir, we can increase or decrease the quality of the images according to our requirements.
\item Use of very high quality graphics increases the running time of the application as such graphics have large file sizes and thus affect the processing time immensely especially during the making of the adobe postscript format file which is further converted to the pdf format after highly compressing the file. In case of the test data used by us, we used 20 MB graphics for covers(8 pages) and 40 MB files for backgrounds(78 pages) and thus ended up with a 2 GB postscript file which was further compressed to a 35 MB PDF file. In another instance, we used 2 MB graphics for both cover pages as well as separator pages. This gave us a 180 MB postscript file which was further compressed to a 3.1 MB pdf file. Thus, size of the graphics plays a major role in the processing time and disk space usage requirements.
\image{0.2}{images/All.png}{All Files Created}
\item postscript file which was further compressed to a 3.1 MB pdf file. Thus, size of the graphics plays a major role in the processing time and disk space usage requirements.
\item The application can also be made light weight by using small sized graphics which
will not only reduce running time but also save time on making them as not much effort will be needed to make them. Application of pictures of students can be done through a simple code snippet that might be included in the main script itself or might be added as a separate utility that has to be run by the user before the main script is run.
\item The time measurements and comparisons have been done in seconds and thus the accuracy of these might vary under different test conditions.
\image{0.3}{images/2.png}{Cover Page}
\end{itemize}